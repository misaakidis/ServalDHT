%This work is licensed by the author, Isaakidis Marios, under the Creative Commons Attribution 3.0 Unported License, in memory of Aaron Swartz. To view a copy of this license, visit http://creativecommons.org/licenses/by/3.0/ or send a letter to Creative Commons, 444 Castro Street, Suite 900, Mountain View, California, 94041, USA.

\documentclass[12pt,a4paper,oneside]{article}
\usepackage{graphicx}
\usepackage{tikz}
\usepackage{hyperref}
\usepackage{nomencl}

\makenomenclature

\begin{document}

%Insert titlepage
\begin{titlepage}

\begin{center}

\newcommand{\HRule}{\rule{\linewidth}{0.5mm}}

% Upper part of the page   
%\textsc{\large DIPLOMA THESIS}\\[1.5cm]


% Title
\HRule \\[0.5cm]
{ \LARGE \bfseries {\huge Service-Aware Networking:} \\[0.2cm] Service-Centric Architectures\\[0.2cm]and the SDN Paradigm}\\[0.5cm]

\HRule \\[1cm]

{\LARGE \bf
Isaakidis Marios\\
}
misaakidis@yahoo.gr

\vfill

% Bottom of the page
{\large
The research and implementation ideas described in\\ this thesis are developed under the advisement of\\ \textbf{Dr. Sirivianos Michael}\\}
michael.sirivianos@cut.ac.cy
\end{center}
 ~\\[2.1cm]
\begin{flushright}
\includegraphics[width=0.25\textwidth]{./cut-logo-2}\\[0.2cm]
{\large
June 2014 \\
Cyprus University of Technology
}
\end{flushright}

\end{titlepage}
\clearpage\null\vfill
\thispagestyle{empty}
\begin{center}
Copyright \copyright \href{https://creativecommons.org/licenses/by/3.0/}{CC-BY 3.0} 2012--\the\year\ Isaakidis Marios\\[0.5cm]
Permission is granted to copy and distribute this document under the terms of the Creative Commons Attribution 3.0 Unported Licesne\ldots.
\end{center}
\clearpage


%%%%%%%%%%%%%%%%%%%%%%%%%%%%%%%%%%%%%%%%%%%%%%%%%%%%%%%%%%%%%%%%%%%%
\newpage
\thispagestyle{empty}
{\Large \bf \noindent Acknowledgments}
\\Here go the acknowledgments


%%%%%%%%%%%%%%%%%%%%%%%%%%%%%%%%%%%%%%%%%%%%%%%%%%%%%%%%%%%%%%%%%%%%
\newpage
\thispagestyle{empty}
{\Large \bf \noindent Abstract}
\\Here goes the final thesis abstract


%%%%%%%%%%%%%%%%%%%%%%%%%%%%%%%%%%%%%%%%%%%%%%%%%%%%%%%%%%%%%%%%%%%%
\newpage
\thispagestyle{empty}
\tableofcontents
%\listoffigures
%\listoftables


%%%%%%%%%%%%%%%%%%%%%%%%%%%%%%%%%%%%%%%%%%%%%%%%%%%%%%%%%%%%%%%%%%%%
\newpage
%This is the first page with a number
\pagestyle{plain}
\setcounter{page}{1}

\renewcommand{\nomname}{Abbreviations}
\addcontentsline{toc}{section}{Abbreviations}
\printnomenclature


%%%%%%%%%%%%%%%%%%%%%%%%%%%%%%%%%%%%%%%%%%%%%%%%%%%%%%%%%%%%%%%%%%%%
\newpage
\begin{center}
{\large {\bf  Service-Aware Networking: Service-Centric Architectures and the SDN Paradigm}\\[0.5cm] by \\[0.5cm] Isaakidis Marios - 2009437805}
 ~\\[0.5cm]
Submitted to the Department of Electrical Engineering, Computer Engineering and Informatics on June 2014, in partial fulfillment of the requirements for the degree of Electrical Engineering, Computer Engineering and Informatics
\end{center}

\vfill

{\Large \bf \noindent Introduction} \\[0.5cm]
\addcontentsline{toc}{section}{Introduction}
The aim of this report is to give a thorough depiction of the current progress in the preparation of ServalDHT, a decentralized system for resolving serviceIDs in the Serval Architecture \cite{Nordstrom2012}.
ServalDHT utilizes Distributed Hash Tables \nomenclature{DHT}{Distributed Hash Table} as a peer-powered DNS \nomenclature{DNS}{Domain Name System}  alternative in order to enable users locate service providers using a human readable service name.
Because of its nature, ServalDHT faces issues of security, agility and robustness in real-world scenarios, and experiments should demonstrate that it confronts them with great success before it can be widely adopted.\\
\indent This report comes as a result of methodical study of existing systems and reasoning on how to propose a solid, grounded on well-known resources yet innovative solution to improve their scalability and adaptability.
First, in sections 1 and 2, it is discussed the general idea of the problems this thesis expects to elucidate, the importance of them and their consequences.
Then, in the following two sections, it is outlined the theoretical background acquired by analyzing extant proposals and researching on relative topics.\\
\indent Finally, in sections 5 and 6 follows a brief introduction on the proposed solution, its main features and how it is going to diminish the inconveniences stated before, along with the expected results of the future implementation and its strain testing.

~\\[0.5cm]
{\large
\noindent Thesis Supervisor: Dr. Sirivianos Michael\\
\noindent Title: Lecturer at CUT's EEIT Department}


%%%%%%%%%%%%%%%%%%%%%%%%%%%%%%%%%%%%%%%%%%%%%%%%%%%%%%%%%%%%%%%%%%%%
\newpage
\section{Problem Definition}
The concept of Internet has radically changed since its first onset, around half a century ago; millions of multi-homed users, possibly moving across networks, are asking for data and services offered by multiple servers, which can be replicated and situated in various geographical locations.
Yet, due to legacy reverse compatibility reasons, bureaucracy obstructions and the compulsion of large scale testing and deployment, only a few modifications managed to consolidate and provide the framework for communicating in the largest computer network.
This situation is leading to erratic band-aids where network administrators and developers overload the existing network abstractions, like the IP \nomenclature{IP}{Internet Protocol} addresses and ports, in order to provide the supplementary functionality needed by a network with dynamic users and where services and data become first-class citizens.\\
\indent In addition, it is observed that the freedom in Internet is menacingly encircled by equivocal organizations trying to be the ones who will win the authorship and control over its content and autonomy.\\
\indent In the subsections following we take a closer look to the problems ServalDHT intents to elucidate divided by their main source.

\subsection{An obsolete network stack}
The network TCP/IP \nomenclature{TCP}{Transmission Control Protocol} stack which is still used today was designed in an era when end hosts where static in specific topological positions, communicating over a sole network interface, accessing services like telnet and ftp.
The problems by this approach start to accumulate even in the lowest layers, and specifically the Network Layer.
\paragraph{Network Layer} The Network Layer is responsible for packet forwarding, including routing through intermediate routers, and it does so using a hierarchical IP addressing scheme.
This bind however of a topological-aware IP address to an interface does not manage well with the notion of mobility, where interfaces are not necessarily tied to a specific network.
Nevertheless, an IP address cannot identify forever a host since after a disconnection, the IP address is renewed to one that was most likely previously used by another interface of another machine.
\pagebreak
\paragraph{Transport Layer} The Transport Layer provides end-to-end communication services for applications within a layered architecture of network components and protocols.
This is achieved by demultiplexing incoming packets to a socket using the five-tuple (remote IP, remote port, local IP, local port, protocol).
Since local IP is tied to a unique interface, support for migration or multi-path traffic over multiple network interfaces has to be implemented individually by the protocol or the above layers.
Never to forget that every time a renewal of IP address occurs the connection has to be reestablished or at least the other end host has to be notified somehow for the new address. Also, without serving any particular reason, the remote IP address and port have to be exposed to the upper layers. \\
\indent For the case of load balancers, every single packet, even from an already established connection, has to pass through them.
This results in a need for dedicated software or hardware, proliferates the demanded computational power, and causes unnecessary ''east-west'' machine-to-machine traffic. In large scale networks with nodes distributed in distant topological locations this can evoke router stretching and increased latency times.
\paragraph{Application Layer}  The Application Layer is an abstraction layer reserved for communication protocols and methods designed for process-to-process communications across an Internet Protocol (IP) computer network.
Because of the overload of IP addresses and ports on lower layers, the Application Layer has to cache them and handle them too.
At the same time, violating the principle of software reuse, each application has to implement from scratch all the logic for the additional functionality of modern Internet (migration, multiple clients support, multihoming, load balancing, mobility etc.), in order to offer it to its users.\\
\indent Another complication in the Application Layer can be detected during the initiation of a connection, and especially during the mapping of a service identifier to an IP address.
As of now, applications must use out-of-band services like DNS and follow preconcerted conventions before the commencement of the connection.
Additionally, by caching the IP address of the service provider instead or re-resolving the service identifier, the service provider is constrained in changing its IP address (in cases of migration, machine or network failure, multihoming etc.), as it will result to the termination of the established connections and a slow failover, considering that some time is needed for the DNS distributed servers to be updated and to respond correctly to the clients.

\newpage
\subsection{The need for Service-Centric Networking}
In the very early Internet, "calling" the IP address of a machine would get you to one of the killer applications of that time, telnet or ftp.
Those services were run by a single machine and could not accept simultaneous users.
However this approach is not common nowadays, when hundreds of users want to search a keyword in their favorite search engine at the same time.
They do not care about the actual location on the map of the service provider, or which of the machines is serving them accessing a distributed database.
Neither the database of a search engine is that small that can be stored in a single hard disk nor a sole machine can respond to all those requests.
Still such services exist and manage well with the always increasing demand. \\
\indent It is only because developers and network administrators are utilizing middleboxes and implementing intermediate systems in order to overcome the deficiencies caused by the superseded network abstractions.
However, this comes with a cost.
Developers have to work with primitive, low-level APIs and to handle many cases of downfalls, needing many costly man-hours, being prone to errors, repeating the same procedure again and again diverging from efficient code writing.
System administrators have to master all those intermediate systems and make them work agreeably.
Routers route packets containing both data and network identifiers without the ability of policy governed delegation.
Replicated service instances run autonomously without a way to directly communicate with each other in a network level.
Master nodes in clusters shoulder the responsibility of the reinstatement over network failures in a wavering manner.
Middleboxes evoke large time delays, they need extra hardware, power, space.
And the list goes on.\\
\indent To sum up, users nowadays want to access a service or to retrieve some data.
The abstraction of a service can suit well any use of Internet anyone can think of; watch a video, send an e-mail, make a phone call, remotely access a distant machine. Unfortunately so far there is no standardized practice for effectively developing and administering services, abandoning developers to create their own mercurial quick fixes, an expensive, time consuming, inclined to mistakes and complex in orchestration solution.

\newpage
\subsection{A unified control and data plane}
Networking is a constantly developing constituent of the computer science and has played a vast role in its necessity and spreading.
Someone would expect that administering networks is a straight-forward and automated task and innovation scenarios can be easily tested and employed.
However, that's not the case\ldots \\
\indent 


Split those two
\\Proprietary technologies, lack of APIs (programmable interfaces) or proper abstractions, non-scalable, inflexible, and troublesome to learn combine administer
\\Complexity because of many discortant developed protocols
\\Stops innovation and agility in network architecture development
\\Does not cope well with mobile users, server virtualization, cloud services
\\Today’s applications access different databases and servers, creating a flurry of “east-west” machine-to-machine traffic before returning data to the end user device in the classic “north-south” traffic pattern.
\\Require device-level management and manual processes (time! money! availability! errors!)
\\resolve newly observed, constantly arising problems in the current Internet?


%%%%%%%%%%%%%%%%%%%%%%%%%%%%%%%%%%%%%%%%%%%%%%%%%%%%%%%%%%%%%%%%%%%%
\newpage
\section{Importance of the Problem}
New solutions copy the current stack so changes should be made as soon as possible
\\Necessary benefits for users (multiplicity and dynamism) and for developers (easy, time and money saving, walk through)
\\Administrators must have a better control over the network
\\Freedom in Internet, especially after social phenomenon (news write much about it, after SOPA PIPA etc)
\\Autonomous networks need this for service resolution


%%%%%%%%%%%%%%%%%%%%%%%%%%%%%%%%%%%%%%%%%%%%%%%%%%%%%%%%%%%%%%%%%%%%
\newpage
\section{Description of Relative Systems}
\subsection{Introduction}
Serval\footnote{More information about the Serval Architecture can be found in the presentation in the Appendix (\ref{sec:servaldhtpres}).} is an end-host stack and a service-centric network architecture, proposed and prototyped by the \href{https://sns.cs.princeton.edu/}{systems and networking group} at \href{https://www.princeton.edu}{Princeton University}, in 2012.

\paragraph{} In the original paper "A Service Access Layer, at Your Service" (2011)\cite{Freedman2011} and later on "Serval: An End-Host Stack for Service-Centric Networking" (2012)\cite{Nordstrom2012}, Nordstr{\"o}m et. al. first decompose the needs of modern networked applications, locate the discordances with the current Network Stack, study previous work and how each of them individually fails to stand as a proper solution, reconsider the current TCP/IP Networking Stack and propose two simple abstractions that can obliterate the legacy problems discussed on Problem Definition\ref{problemdefinition}.\\
\indent Furthermore they introduce a formally-verified end-to-end connection control protocol (ECCP) 

\subsection{Serval Architecture in a few words}


\subsection{Proposed abstractions}



\subsection{Serval Network Stack}
\subsubsection{Service Controller}
\subsubsection{Service Access Layer}
\subsection{Service Resolution in the Serval architecture}

\subsection{Application portability and incremental deployment}
- A translator can be used but: (ADD SCHEMA)
works as an intermediate
slows down performance
apps still work with the traditional apis
 -- adds complexity, translator will have to support many versions/protocols and error proof for them
 -- reimplements the functionality already working on SAL


\subsection{Migrations and incremental deployment}
With Serval being actively under development, it is time to discuss the deployment approaches that could guide us to something that has never happened before; the wide adoption of a new network stack.
Above all, benchmarks prove that tapping Serval in large scale networks as well as datacenters offers a wide range of new functionality in speeds comparable to the original TCP/IP stack ones.\\
\indent A smooth transition to a new architecture requires two things: first that current hardware and intermediary devices are compatible or at least do not interfere with the new packet headers, and that applications are utilizing the late interfaces and are able to dissect and synthesize those packets.

\subsubsection{Legacy Hardware}
SAL's position just on top of the network layer makes it translucent to networking equipment such as hubs, switches and routers, since they are messing up with the headers of up to the network layer.
At this level, hardware is responsible only for delivering the packets to the right destination, the way they have been doing so far.\\
\indent Serval on the contrary is not immune to stateful packet inspection\nomenclature{SPI}{Stateful Packet Inspection} and deep packet inspection\nomenclature{DPI}{Deep Packet Inspection}.
Intermediaries who access the headers of the transport or above layers will have a hard time dissecting a minimum 32 extra bytes following the network layer.
The use of NAT-based agents though, such as load balancers, can be obscured due to the late binding on serviceIDs.
In any way, operation behind legacy networks and NATs can be achieved via UDP encapsulation.

\subsubsection{Modified Programming Interfaces}
Unlike other proposals which can be either integrated in programs as libraries or provide abstractions by overloading identifiers such as ports, Serval's kernel module implementation requires applications to be modified in order to use its active sockets API.\\
\indent In other words, a minimum port of an existing applications would require to include and link to \textless libserval/serval.h\textgreater ~and \textless netinet/serval.h\textgreater ~libraries, set socket family to AF\_SERVAL and substitute system calls to the socket layer such as connect, bind, accept, send etc to use serviceIDs.
Minor modifications might be needed, since new identifiers require different size of bytes to be allocated in memory and so on.\\
\indent The complexity of porting an existing application to Serval depends on how neatly is the connection module isolated.
In general, applications that support various protocols are easier to be ported, since connectivity functions are already decoupled from the program logic and can be replicated to support new APIs.
Large applications, with thousands lines of code,\nomenclature{loc}{lines of code} like Mozilla Firefox require modifications in around just 70 lines of code.
Design patterns like singleton and factory noticeably indicate which parts of the source code should be altered.\\
\indent Nevertheless, applications can simultaneously support multiple protocols and stacks, according to the requests of the other end.
This will be a great advantage in the transitory period of large scale deployment.\\
\indent On the other hand, unexpected runtime results might be confronted due to overlapping of features offered by SAL and reimplemented in the program.
For instance service load balancing at SAL, an inherent component of Serval, might muddle with the application-specific way of allotting requests.
In such cases if possible one of the methods should dominate the final decision.
Either a unique serviceID of an allocated serviceID block --as described in the Hierarchical Resolution approach-- should be used for each instance, and let load-balancing be made by application's functions.
Or load-balancing logic within the application should be removed when a socket's family is AF\_SERVAL and be managed according to service-level rules defined in SAL.\\
\indent We are discussing the lessons learned from the port of nginx\footnote{Nginx [engine-ex] is an Open Source HTTP, reverse proxy and mail proxy server\\ \url{http://nginx.org/}.} to Serval in the following section.


\subsubsection{nginx port to Serval}

Lessons learned from nginx.
There are many 

- Requires to rebuild the whole project with the new abstractions (Serval active sockets) and include Serval libraries

- Depends on the independence of the connectivity modules
if patterns like singleton then porting is easier
calls in many files then complexity arises

- Extra implemented functionality can cause unexpected runtime results
example: 2 levels of loadbalancing


\subsubsection{Incremental Deployment with Serval translator}

- A translator can be used but: (ADD SCHEMA)
works as an intermediate
slows down performance
apps still work with the traditional apis
 -- adds complexity, translator will have to support many versions/protocols and error proof for them
 -- reimplements the functionality already working on SAL

\subsection{Profiling the Serval prototype implementation}
Among the admirable headliners of Serval is the working prototype version of the proposed architecture.
In more than 28000 lines of code\footnote{Source code is open source and can be found in their public repository\\ \url{https://github.com/princeton-sns/serval/}.} covering functionality of the Service Access Layer (both in userlevel operation and as a Linux kernel module), bindings for multiple programming languages, a translator, libraries and examples for writing Serval compatible applications, and with a reported throughput comparable to the unmodified TCP/IP stack, it is clearly showcased the feasibility of the solution.\\
\indent In this section we are profiling the prototype in regard to the following parameters:
\begin{enumerate}
  \item Memory Management
  \item CPU Instructions and Cycles
  \item System Calls times
  \item Execution time needed for the completion of a numbered iteration of requests
  \item The sum of requests that can be satisfied within a given timeframe
\end{enumerate}
Then we will be graphically presenting the results juxtaposed to the measurements of the unchanged TCP/IP stack and the AF\_INET family. The first four tests have not been published before.
\paragraph{} Output was obtained on a ...... machine running Ubuntu 11.04 (Natty Narwahl) kernel version 2.6.38-16-generic (rebuilt with debug symbols). Serval was built with --enable-debug option for the first three tests.\\
For the measurements we ported the nginx\footnote{Nginx [engine-ex] is an Open Source HTTP,reverse proxy and mail proxy server\\ \url{http://nginx.org/}.} web server to Serval architecture, integrating the [nginx\_1.2.9\_serval\_fqdn] branch commits into the build procedure. Also, we implemented a simple HTTP client which supports both AF\_INET and the AF\_SERVAL socket families, depending on the options passed during the call. Source code of both can be found in the Appendix (\ref{sec:sourcecode}).


\iffalse
gprof
perf
google-profile tools
1) memory (oprofile)
2) CPU cycles (callgrind)
3) system calls time (strace)
4) timed execution of 1000 times
5) requests per second
6) Number of packers per request, bytes sent, packet structure
\fi

TODO: Complete this section, maybe add HIP, Chord and OpenFlow


%%%%%%%%%%%%%%%%%%%%%%%%%%%%%%%%%%%%%%%%%%%%%%%%%%%%%%%%%%%%%%%%%%%%
\newpage
\section{Review of Relative Bibliography}
ServalDHT is a multifaceted architecture that combines ideas from a wide spectrum of topics, including but no limited to Large Scale Network Architectures, Network Protocol Layers, Service-Centric Networking, Software Defined Networking, Distributed Hash Table algorithms and security issues of their various implementations, Peer-To-Peer Lookup Services as a replacement to legacy DNS etc.
Therefore references should contain an adequate number of publications on all those themes.\\
\indent The publications that have been used so far as a source of information follow in section 7.
Brief summaries can be found in Appendix at the end of the report.


%%%%%%%%%%%%%%%%%%%%%%%%%%%%%%%%%%%%%%%%%%%%%%%%%%%%%%%%%%%%%%%%%%%%
\newpage
\section{Rethinking the Internet experience}
%I HAVE A DREAM
Cloud computing
\\Modularity
\\anonymity, privacy
\\authentication, accountability, encryption, security
\\no middlewares
\\TCP/IP stack data, controller functionality
\\subnetworks
\\mobility and multihoming
\\independent from organizations


%%%%%%%%%%%%%%%%%%%%%%%%%%%%%%%%%%%%%%%%%%%%%%%%%%%%%%%%%%%%%%%%%%%%
\newpage
\section{Service-centric architectures and Software Defined Networking}


%%%%%%%%%%%%%%%%%%%%%%%%%%%%%%%%%%%%%%%%%%%%%%%%%%%%%%%%%%%%%%%%%%%%
\newpage
\section{ServalSDN and cloud networks}


%%%%%%%%%%%%%%%%%%%%%%%%%%%%%%%%%%%%%%%%%%%%%%%%%%%%%%%%%%%%%%%%%%%%
\newpage
\section{Introduction to the Proposed Solution}
User Serval as it is, but with a Distributed Service Resolution Service
\\Based on DHT algorithms
\\Run by tier-1 and ISPs, or by users in Autonomous networks
\\Secure identifiers
\\Flat namespace
\\Gets serviceID and
\\- either returns the (IP,Port) back to the client
\\- or forwards the packet directly to the service provider
\\- caches the (serviceID, IP) tuple for future use
\\Incrementally deployable and backwards compatible
\\written in C, running in the user space
\\accept service registration, checks HIP and inform the relative nodes
\\About the Service Controller:
\\will be running as a daemon in the userspace
\\will communicate with ServalDHT to resolve serviceIDs (hashed service names)
\\will enable delegation, load balancing etc as modules (maybe a configuration file?)
\\Draw a FSM for the SRS \nomenclature{SRS}{Service Resolution Service} (http://madebyevan.com/fsm/)


%%%%%%%%%%%%%%%%%%%%%%%%%%%%%%%%%%%%%%%%%%%%%%%%%%%%%%%%%%%%%%%%%%%%
\newpage
\section{Expected Results}
1. Implement Service Controller (compatible with OpenFlow)
\\2. Implement ServalDHT SRS
\\3. Results of deployment in PlanetLab
\\4. Start preparing an RFC?
\\See how DHTs can replace hierarchical DNS?


%%%%%%%%%%%%%%%%%%%%%%%%%%%%%%%%%%%%%%%%%%%%%%%%%%%%%%%%%%%%%%%%%%%%
\newpage
\section{Bibliography}
\nocite{*}
\bibliographystyle{plain}
\renewcommand{\refname}{}
\bibliography{bibliography}


%%%%%%%%%%%%%%%%%%%%%%%%%%%%%%%%%%%%%%%%%%%%%%%%%%%%%%%%%%%%%%%%%%%%
\newpage
\thispagestyle{empty}
{\Huge \bf \noindent APPENDIX}
\addcontentsline{toc}{section}{APPENDIX}
\newpage

\end{document}