\newpage
\section{The Serval Networking Architecture}
\subsection{Introduction}
Serval\footnote{More information about the Serval Architecture can be found in the presentation in the Appendix (\ref{sec:servaldhtpres}).} is an end-host stack evolving into a service-centric network architecture, proposed and prototyped by the \href{https://sns.cs.princeton.edu/}{systems and networking group} at \href{https://www.princeton.edu}{Princeton University}, in 2012.

\paragraph{} In the original paper "A Service Access Layer, at Your Service" (2011)\cite{Freedman2011} and later on "Serval: An End-Host Stack for Service-Centric Networking" (2012)\cite{Nordstrom2012}, Freedman, Nordstr{\"o}m et. al. first decompose the needs of modern networked applications, locate the discordances with the current Network Stack, study previous work and how each of them individually fails to stand as a proper solution, rethink the current TCP/IP Networking Stack and propose two simple abstractions that can obliterate the legacy problems discussed on Problem Definition (\ref{problemdefinition}).\\
\indent Furthermore they investigate how those abstractions fit in a new 3.5 layer, the Service Access Layer (SAL)\nomenclature{SAL}{Service Access Layer}, and emphasize on the clean service-level data/control plane separation it imposes.
Additionally, they review a formally-verified end-to-end connection control protocol (ECCP) \nomenclature{ECCP}{End-to-End Connection Control Protocol} [TODO], which elevates services to first-class citizens.
After all, they focus on the SAL prototype and the lessons learned from building it.

\subsection{Proposed abstractions}
As will be discussed later on at the section "Review of Relative Bibliography" (\ref{sec:reviewbibliography}), various abstractions have been proposed in order to support the concept of Service-Centric networking.
Deciding the right abstractions premises first the detailed understanding of where a problem resides, and then the genuine intuition for conceiving the simple yet powerful generalizations that better conform with the current and future needs.
Those abstractions should be transparent enough for legacy ports and hot swapping with the prevalent systems, in the meantime powerful to break ground for future advancements.\\
\indent We appraise the Serval abstractions of \emph{Service Names} and \emph{Flow Identifiers} for being a great solution on the overloading of identifiers within and across the present TCP/IP stack.
Also, for the tight fit of those abstractions in the separation of data and control plane.
Finally, it is remarkable how the adoption of those abstractions requires minimal if no modifications at all in the infrastructure that powers networks today.

\subsubsection{Service Names}
The main idea behind Service-Centric Networking is that nowadays network users request services by their names and are completely agnostic of the underlying procedures.
Hereby an abstraction is needed for identifying a service, resolving and of course effectively routing a request to an instance of that service, according to specific rules.
For that reason Serval introduces service names, or \emph{serviceIDs}.\\
\indent ServiceIDs are large (256 bit long) strings, that correspond to a service, or a group of services.
Part of Serval packets as a Service Access Extension header in the synchronization process (SYN and SYN-ACK flags in ECCP state machine figure \ref{fig:ECCP_st}), serviceIDs assist in resolution and service-level routing.
Furthermore they are used as persistent, global identifiers for easier handling of replicated services, virtual machine migrations or mobile hosts, when there is a need for end-point signaling.\\
\indent What is unique about Serval's abstraction of service identifiers is that since they are managed by the SAL, they are positioned below the transport layer.
Jointly with Serval active sockets, the abstraction of service names obscures host identifiers (IP and Port) from the application layer, but stills offers powerful, familiar APIs\nomenclature{API}{Application Programming Interface} to networked application developers.
\indent service granularity (block allocation) \\
\indent To sum up, in an elegant way serviceIDs promote services to the principal entities in the network, by giving applications intermediate access to the service control plane and a
And it manages to do so while hiding unneeded abstractions, such as the $<IP, port>$ tuple.

\subsubsection{End-host flow identifiers}

\subsection{Service Resolution}
Those service names in a deeper level get mapped to 
human readable strings similar to domain names, for identifying services.
regardless it's geographical position, the port it is bound, 

\subsubsection{Hierarchical Resolution}
Relative to today's domain name system, those serviceIDs can be mapped to human-readable strings, much like URIs.

\subsubsection{Flat Resolution Schemes}
Another approach suggests that serviceIDs are calculated by a hash function on a service prefix and an application-level instance unique key.
This practice of a \emph{semantic-free, flat} namespace better supports networks without a hierarchical structure, like datacenter and Wide Area networks.
By semantic-free\cite{Walfisha2004} reference we mean that by default 

\subsection{Service-Level Routing} 
No need for deep packet inspection in load balancers


\subsubsection{Late binding}

\subsection{Serval Network Stack}
Extra points go to Serval for being transport-protocol agnostic.
This means that developers may choose to use either TCP, UDP\nomenclature{UDP}{User Datagram Protocol}, ATP\nomenclature{ATP}{AppleTalk Transaction Protocol}, FCP\nomenclature{FCP}{Fibre Channel Protocol} or actually any transport protocol, since it is supported by SAL, without modifying the source code of their application.
This opens new windows for experimentation and innovation.

\subsubsection{Service Controller}

\subsubsection{Service Access Layer}

\subsubsection{Serval Packets Structure}

\subsection{Migrations and incremental deployment}
With Serval being actively under development, it is time to discuss the deployment approaches that could guide us to something that has never happened before; the wide adoption of a new network stack.
Above all, benchmarks prove that tapping Serval in large scale networks as well as datacenters offers a wide range of new functionality in speeds comparable to the original TCP/IP stack ones.\\
\indent A smooth transition to a new architecture requires two things: first that current hardware and intermediary devices are compatible or at least do not interfere with the new packet headers, and that applications are utilizing the late interfaces and are able to dissect and synthesize those packets.

\subsubsection{Legacy Hardware}
SAL's position just on top of the network layer makes it translucent to networking equipment such as hubs, switches and routers, since they are messing up with the headers of up to the network layer.
At this level, hardware is responsible only for delivering the packets to the right destination, the way they have been doing so far.\\
\indent Serval on the contrary is not immune to stateful packet inspection\nomenclature{SPI}{Stateful Packet Inspection} and deep packet inspection\nomenclature{DPI}{Deep Packet Inspection}.
Intermediaries who access the headers of the transport or above layers will have a hard time dissecting a minimum 32 extra bytes following the network layer.
The use of NAT-based agents though, such as load balancers, can be obscured due to the late binding on serviceIDs.
In any way, operation behind legacy networks and NATs can be achieved via UDP encapsulation.

\subsubsection{Modified Programming Interfaces}
Unlike other proposals which can be either integrated in programs as libraries or provide abstractions by overloading identifiers such as ports, Serval's kernel module implementation requires applications to be modified in order to use its active sockets API.\\
\indent In other words, a minimum port of an existing applications would require to include and link to \textless libserval/serval.h\textgreater ~and \textless netinet/serval.h\textgreater ~libraries, set socket family to AF\_SERVAL and substitute system calls to the socket layer such as connect, bind, accept, send etc to use serviceIDs.
Minor modifications might be needed, since new identifiers require different size of bytes to be allocated in memory and so on.\\
\indent The complexity of porting an existing application to Serval depends on how neatly is the connection module isolated.
In general, applications that support various protocols are easier to be ported, since connectivity functions are already decoupled from the program logic and can be replicated to support new APIs.
Large applications, with thousands lines of code,\nomenclature{loc}{lines of code} like Mozilla Firefox require modifications in around just 70 lines of code.
Design patterns like singleton and factory noticeably indicate which parts of the source code should be altered.\\
\indent Nevertheless, applications can simultaneously support multiple protocols and stacks, according to the requests of the other end.
This will be a great advantage in the transitory period of large scale deployment.\\
\indent On the other hand, unexpected runtime results might be confronted due to overlapping of features offered by SAL and reimplemented in the program.
For instance service load balancing at SAL, an inherent component of Serval, might muddle with the application-specific way of allotting requests.
In such cases if possible one of the methods should dominate the final decision.
Either a unique serviceID of an allocated serviceID block --as described in the Hierarchical Resolution approach-- should be used for each instance, and let load-balancing be made by application's functions.
Or load-balancing logic within the application should be removed when a socket's family is AF\_SERVAL and be managed according to service-level rules defined in SAL.\\
\indent We are discussing the lessons learned from the port of nginx\footnote{Nginx [engine-ex] is an Open Source HTTP, reverse proxy and mail proxy server\\ \url{http://nginx.org/}.} to Serval in the following section.


\subsubsection{nginx port to Serval}

Lessons learned from nginx.
There are many 

- Requires to rebuild the whole project with the new abstractions (Serval active sockets) and include Serval libraries

- Depends on the independence of the connectivity modules
if patterns like singleton then porting is easier
calls in many files then complexity arises

- Extra implemented functionality can cause unexpected runtime results
example: 2 levels of loadbalancing


\subsubsection{Incremental Deployment with Serval translator}

- A translator can be used but: (ADD SCHEMA)
works as an intermediate
slows down performance
apps still work with the traditional apis
 -- adds complexity, translator will have to support many versions/protocols and error proof for them
 -- reimplements the functionality already working on SAL

\subsection{Profiling the Serval prototype implementation}
Among the admirable headliners of Serval is the working prototype version of the proposed architecture.
In more than 28000 lines of code\footnote{Serval is an Open Source code, hosted at a public repository\\ \url{https://github.com/princeton-sns/serval/}.} covering functionality of the Service Access Layer (both in userlevel operation and as a Linux kernel module), bindings for multiple programming languages, a translator, libraries and examples for writing Serval compatible applications, and with a reported throughput comparable to the unmodified TCP/IP stack, it is clearly showcased the feasibility of the solution.\\
\indent In this section we are profiling the prototype in regard to the following parameters:
\begin{enumerate}
  \item Memory Management
  \item CPU Instructions and Cycles
  \item System Calls times
  \item Execution time needed for the completion of a numbered iteration of requests
  \item The sum of requests that can be satisfied within a given timeframe
\end{enumerate}
Then we will be graphically presenting the results juxtaposed to the measurements of the unchanged TCP/IP stack and the AF\_INET family. The first four tests have not been published before.

\paragraph{} Output was obtained on a HP Prodesk 600 G1 --Intel(R) Core(TM) i5-4570 @ 3.20GHZ, 4GB RAM-- machine running Ubuntu 11.04 (Natty Narwahl) kernel version 2.6.38-16-generic (rebuilt with debug symbols). Serval was built with \texttt{-{}-}enable-debug option for the first three tests.\\
\indent For the measurements we ported the nginx web server to Serval architecture, integrating the [nginx\_1.2.9\_serval\_fqdn] branch commits from \mbox{sns/Serval} repository into the build procedure.
Also, we implemented a simple HTTP client which supports both AF\_INET and the AF\_SERVAL socket families, depending on the options passed during the call.
Source code of both can be found in the Appendix (\ref{sec:sourcecode}).

\iffalse
gprof
perf
google-profile tools
1) memory (oprofile)
2) CPU cycles (callgrind)
3) system calls time (strace)
4) timed execution of 1000 times
5) requests per second
6) Number of packers per request, bytes sent, packet structure
\fi