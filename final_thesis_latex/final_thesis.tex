%This work is licensed by the author, Isaakidis Marios, under the Creative Commons Attribution 3.0 Unported License, in memory of Aaron Swartz. To view a copy of this license, visit http://creativecommons.org/licenses/by/3.0/ or send a letter to Creative Commons, 444 Castro Street, Suite 900, Mountain View, California, 94041, USA.

% Export a final_thesis.pdf:
% make clean ; make final_thesis.pdf

\documentclass[12pt,a4paper,oneside]{article}
\usepackage[usenames,dvipsnames]{color}
\usepackage{graphicx}
\usepackage{tikz}
\usepackage[hyphens]{url}
\usepackage{hyperref}
\usepackage{nomencl}
\usepackage{pdfpages}
\usepackage{listings}
\usepackage{caption}
\usepackage{booktabs}
\usepackage{float}
\usepackage{tabularx}
\usepackage{xr}

\def\tabularxcolumn#1{m{#1}}
\externaldocument{*}

\lstdefinelanguage{diff}{
  morecomment=[f][\color{Blue}]{@@},     % group identifier
  morecomment=[f][\color{Red}]-,         % deleted lines 
  morecomment=[f][\color{Green}]+,       % added lines
  morecomment=[f][\color{Magenta}]{---}, % Diff header lines (must appear after +,-)
  morecomment=[f][\color{Magenta}]{+++},
  breaklines=true,breakindent=0pt,breakatwhitespace=false,
  postbreak=\raisebox{0ex}[0ex][0ex]{\ensuremath{\color{black}\hookrightarrow\space}},
  basicstyle=\ttfamily\small
}

\lstdefinelanguage{lua}
% Written by Roland Smith <rsmith@xs4all.nl> and hereby placed in the public domain. 
  {morekeywords={and,break,do,else,elseif,end,false,for,function,if,in,local,
     nil,not,or,repeat,return,then,true,until,while},
   morekeywords={[2]arg,assert,collectgarbage,dofile,error,_G,getfenv,
     getmetatable,ipairs,load,loadfile,loadstring,next,pairs,pcall,print,
     rawequal,rawget,rawset,select,setfenv,setmetatable,tonumber,tostring,
     type,unpack,_VERSION,xpcall},
   morekeywords={[2]coroutine.create,coroutine.resume,coroutine.running,
     coroutine.status,coroutine.wrap,coroutine.yield},
   morekeywords={[2]module,require,package.cpath,package.load,package.loaded,
     package.loaders,package.loadlib,package.path,package.preload,
     package.seeall},
   morekeywords={[2]string.byte,string.char,string.dump,string.find,
     string.format,string.gmatch,string.gsub,string.len,string.lower,
     string.match,string.rep,string.reverse,string.sub,string.upper},
   morekeywords={[2]table.concat,table.insert,table.maxn,table.remove,
   table.sort},
   morekeywords={[2]math.abs,math.acos,math.asin,math.atan,math.atan2,
     math.ceil,math.cos,math.cosh,math.deg,math.exp,math.floor,math.fmod,
     math.frexp,math.huge,math.ldexp,math.log,math.log10,math.max,math.min,
     math.modf,math.pi,math.pow,math.rad,math.random,math.randomseed,math.sin,
     math.sinh,math.sqrt,math.tan,math.tanh},
   morekeywords={[2]io.close,io.flush,io.input,io.lines,io.open,io.output,
     io.popen,io.read,io.tmpfile,io.type,io.write,file:close,file:flush,
     file:lines,file:read,file:seek,file:setvbuf,file:write},
   morekeywords={[2]os.clock,os.date,os.difftime,os.execute,os.exit,os.getenv,
     os.remove,os.rename,os.setlocale,os.time,os.tmpname},
   sensitive=true,
   morecomment=[l][\color{Magenta}]{--},
   morecomment=[s][\color{Magenta}]{--[[}{]]},
   morestring=[b][\color{Blue}]",
   morestring=[d][\color{Blue}]',
   breaklines=true,breakindent=0pt,breakatwhitespace=false,
   postbreak=\raisebox{0ex}[0ex][0ex]{\ensuremath{\color{black}\hookrightarrow\space}},
   basicstyle=\ttfamily\small
}

\lstdefinestyle{BashColor}
{language=bash,
  keywordstyle=\color{Blue},
  basicstyle=\ttfamily\small,
  keywordstyle=[2]{\color{Red}},
  literate={\$}{{\textcolor{Red}{\$}}}1 
         {:}{{\textcolor{Red}{:}}}1
         {~}{{\textcolor{Red}{\textasciitilde}}}1,
  breaklines=true,breakindent=0pt,breakatwhitespace=true,
  postbreak=\raisebox{0ex}[0ex][0ex]{\ensuremath{\color{black}\hookrightarrow\space}},
}

\lstdefinestyle{CColor}
{language=C,
  keywordstyle=\color{Blue},
  basicstyle=\ttfamily\small,
  keywordstyle=\color{Blue},
  stringstyle=\color{Red},
                commentstyle=\color{Green},
                morecomment=[l][\color{Magenta}]{\#},
  breaklines=true,breakindent=0pt,breakatwhitespace=false,
  postbreak=\raisebox{0ex}[0ex][0ex]{\ensuremath{\color{black}\hookrightarrow\space}},
}
\makenomenclature

\begin{document}

\newcommand{\chapterpage}[2]{
	\newpage
	\thispagestyle{empty}
	\phantomsection
	\addcontentsline{toc}{part}{#1}
	{\Huge \bf \noindent #2}
	\newpage
}

%%%%%%%%%%%%%%%%%%%%%%%%%%%%%%%%%%%%%%%%%%%%%%%%%%%%%%%%%%%%%%%%%%%%
% Titlepages
\pagenumbering{roman}

\iffalse 
\includepdf[pages={-}]{tepak_titlepage.pdf}
\fi

\pdfbookmark[0]{Title}{title}
\begin{titlepage}

\begin{center}

\newcommand{\HRule}{\rule{\linewidth}{0.5mm}}

% Upper part of the page   
%\textsc{\large DIPLOMA THESIS}\\[1.5cm]


% Title
\HRule \\[0.5cm]
{ \LARGE \bfseries {\huge Service-Aware Networking:} \\[0.2cm] Service-Centric Architectures\\[0.2cm]and the SDN Paradigm}\\[0.5cm]

\HRule \\[1cm]

{\LARGE \bf
Isaakidis Marios\\
}
misaakidis@yahoo.gr

\vfill

% Bottom of the page
{\large
The research and implementation ideas described in\\ this thesis are developed under the advisement of\\ \textbf{Dr. Sirivianos Michael}\\}
michael.sirivianos@cut.ac.cy
\end{center}
 ~\\[2.1cm]
\begin{flushright}
\includegraphics[width=0.25\textwidth]{./cut-logo-2}\\[0.2cm]
{\large
June 2014 \\
Cyprus University of Technology
}
\end{flushright}

\end{titlepage}
\newpage
\pagestyle{empty}
\mbox{}

%%%%%%%%%%%%%%%%%%%%%%%%%%%%%%%%%%%%%%%%%%%%%%%%%%%%%%%%%%%%%%%%%%%%
% Copyright
\newpage
\setcounter{page}{2}
\clearpage\null\vfill
\thispagestyle{empty}
\begin{center}
Copyright \copyright \href{https://creativecommons.org/licenses/by/3.0/}{CC-BY 3.0} 2012--\the\year\ Isaakidis Marios\\[0.5cm]
Permission is granted to copy and distribute this document under the terms of the Creative Commons Attribution 3.0 Unported Licesne\ldots.
\end{center}
\clearpage

%%%%%%%%%%%%%%%%%%%%%%%%%%%%%%%%%%%%%%%%%%%%%%%%%%%%%%%%%%%%%%%%%%%%
% Acknowledgements
\newpage
\pagestyle{plain}
\pdfbookmark[0]{Acknowledgements}{acks}
{\Large \bf \noindent Acknowledgements} \\[0.5cm]
Same time last year, I was completely lost.
It was only because of the people who came across my path that this journey has come to an end.

\paragraph{}
This is the minimum tribute I could pay to them; friends, colleagues and advisors towards who I feel only sincere gratitude and will always respect.

\paragraph{}
Foremost, I owe a very important debt to Dr. Sirivianos Michael, advisor and mentor of this thesis and many other aspects of life.
I am thankful for your patience and unrestrictive guidance.
It is fulfilling to find a person you can look up to.

\paragraph{}
Likewise, I truly appreciate the feedback offered by Erik Nordstrom from Princeton SNS Lab, when encountering problems with the internals of the Serval architecture.

\paragraph{}
I would also like to particularly thank Bernard Van De Walle, Product Manager at Nuage Networks, and Erik Neel, head IPD QA Antwerp at Alcatel-Lucent, for their insightful feedback and for helping me realize the huge potential of network virtualization in datacenters.
Service-Aware networking is in a great level a brainchild of your advice.

\paragraph{}
Furthermore, I am grateful to Kosiaris Alex and the team of grnet (Greek Research \& Technology Network), for willingly offering okeanos cloud service as an experiments testbench.

\paragraph{}
Moreover, I have received generous support from Apache Cloudstack developers, as well as by "anonymous" contributors of the Open Source movement in general, who have created such amazing projects as Serval, nginx, Linux, perf, git and so many others.

\paragraph{}
My intellectual debt is to Korina Patelis, fellow hacktivist and Internet pioneer, since her intuition that users will not always want to connect to the big-mother Internet, inspired a whole new chapter on the DHT-based resolution of service names.\\
\indent For the same reason to Komaitis Konstantinos, Policy Advisor at the Internet Society, for his instructive publications and for accepting to discuss on service resolution approaches in regard to today's DNS.

\paragraph{}
The stimulating discussions with Papadopoulos Fragkiskos, Mathieu Goessens, Marios Fotiou and others have broaden the horizons of this thesis.

\paragraph{}
It's time to thank my friends who have been close this whole time, among them Achilleas, Avgoustis, Christos, Chrystalleni, Desi, George, Georgia, Giannis, Kostas, Maria, Marialena, Michalis, Nicholas, Nick, Olia, Rodoula, Vana and the Space Apps and \#hack66 peers.

\paragraph{}
Nevertheless, I mean the deepest gratitude to my parents, Parthenios and Georgia, for their endless love and support. And of course to my dear sister Dimitra, with whom we will be graduating the same day.\\[0.8cm]


\begin{flushright}
\emph{I would like to express my heartfelt appreciation to all of you,\\
without your constant encouragement this dissertation\\
would not have been possible.}
\end{flushright}

%%%%%%%%%%%%%%%%%%%%%%%%%%%%%%%%%%%%%%%%%%%%%%%%%%%%%%%%%%%%%%%%%%%%
% Abstract
\newpage
\thispagestyle{empty}
\pdfbookmark[0]{Abstract}{abstract}
{\Large \bf \noindent Abstract} \\[0.13cm]

\noindent The concept of the Internet has radically changed since its first onset, around half a century ago; millions of multi-homed users, often moving across networks, are asking for data and services offered by multiple servers, which can be replicated and situated in various geographical locations.
Yet only a few modifications managed to consolidate and provide the framework for communicating in the largest computer network.

This situation is leading to erratic band-aids where network administrators and developers overload the existing network abstractions or resort to middleware, in order to provide the supplementary functionality needed by a network where services and data become first-class citizens.

In this thesis we are introducing the approach of Service-Aware Networking, a consolidation of Service-Centric abstractions and the Software Defined Networking (SDN)\nomenclature{SDN}{Software Defined Networking} paradigm.

Starting with an explanation of the principles behind Service-Centric networking, we are focusing on the Serval architecture along with its functional prototype.
Results of benchmarks are presented juxtaposed to the measurements of the unmodified TCP/IP stack.

Finally we are suggesting that Software Defined Networking could benefit from Serval's service-level data/control plane separation and enable services running in – possibly distributed – datacenters to automatically, and according to the rights they have been granted, manipulate virtual networks to better utilize the underlying network infrastructure, conforming to their dynamic needs.\\[0.1cm]

\noindent The project is Open Source and can be found at\\ 
\noindent \href{https://github.com/misaakidis/ServalDHT}{https://github.com/misaakidis/ServalDHT}


%%%%%%%%%%%%%%%%%%%%%%%%%%%%%%%%%%%%%%%%%%%%%%%%%%%%%%%%%%%%%%%%%%%%
% Tables
\newpage
\pdfbookmark[0]{Tables and Lists}{tal}
\pdfbookmark[1]{Table of Contents}{toc}
\tableofcontents

\newpage
\phantomsection
\addcontentsline{toc}{section}{List of Figures}
\listoffigures

\newpage
\phantomsection
\addcontentsline{toc}{section}{List of Tables}
\listoftables

\newpage
\renewcommand{\nomname}{Abbreviations}
\phantomsection
\addcontentsline{toc}{section}{Abbreviations}
\printnomenclature


%%%%%%%%%%%%%%%%%%%%%%%%%%%%%%%%%%%%%%%%%%%%%%%%%%%%%%%%%%%%%%%%%%%%
% Introduction
\newpage
\pagenumbering{arabic}
\setcounter{page}{1}
\begin{center}
\phantomsection
\addcontentsline{toc}{part}{Introduction}
{\large {\bf  Service-Aware Networking: Service-Centric Architectures and the SDN Paradigm}\\[0.5cm] by \\[0.5cm] Isaakidis Marios - 2009437805}
 ~\\[0.5cm]
Submitted to the Department of Electrical Engineering, Computer Engineering and Informatics on June 2014, in partial fulfillment of the requirements for the degree of Electrical Engineering, Computer Engineering and Informatics
\end{center}

\vfill

{\Large \bf \noindent Introduction} \\[0.5cm]
The aim of this report is to give a thorough depiction of the current progress in the preparation of ServalDHT, a decentralized system for resolving serviceIDs in the Serval Architecture \cite{Nordstrom2012}.
ServalDHT utilizes Distributed Hash Tables \nomenclature{DHT}{Distributed Hash Table} as a peer-powered DNS \nomenclature{DNS}{Domain Name System}  alternative in order to enable users locate service providers using a human readable service name.
Because of its nature, ServalDHT faces issues of security, agility and robustness in real-world scenarios, and experiments should demonstrate that it confronts them with great success before it can be widely adopted.\\
\indent This report comes as a result of methodical study of existing systems and reasoning on how to propose a solid, grounded on well-known resources yet innovative solution to improve their scalability and adaptability.
First, in sections 1 and 2, it is discussed the general idea of the problems this thesis expects to elucidate, the importance of them and their consequences.
Then, in the following two sections, it is outlined the theoretical background acquired by analyzing extant proposals and researching on relative topics.\\
\indent Finally, in sections 5 and 6 follows a brief introduction on the proposed solution, its main features and how it is going to diminish the inconveniences stated before, along with the expected results of the future implementation and its strain testing.

~\\[0.5cm]
{\large
\noindent Thesis Supervisor: Dr. Sirivianos Michael\\
\noindent Title: Lecturer at CUT's EEIT Department}


%%%%%%%%%%%%%%%%%%%%%%%%%%%%%%%%%%%%%%%%%%%%%%%%%%%%%%%%%%%%%%%%%%%%
% Chapter: Problem Definition
\chapterpage{Problem Definition}{PROBLEM DEFINITION}

\newpage
\section{Defining the Problem}
\label{problemdefinition}
The concept of Internet has radically changed since its first onset, around half a century ago; millions of multi-homed users, possibly moving across networks, are asking for data and services offered by multiple servers, which can be replicated and situated in various geographical locations.
Yet, due to legacy reverse compatibility reasons, bureaucracy obstructions and the compulsion of large scale testing and deployment, only a few modifications managed to consolidate and provide the framework for communicating in the largest computer network.
This situation is leading to erratic band-aids where network administrators and developers overload the existing network abstractions, like the IP \nomenclature{IP}{Internet Protocol} addresses and ports, in order to provide the supplementary functionality needed by a network with dynamic users and where services and data become first-class primitives.

In addition, it is observed that the freedom in Internet is menacingly encircled by equivocal organizations trying to be the ones who will win the authorship and control over its content and autonomy.

In the subsections following we take a closer look to the problems Service-Aware networking intents to elucidate divided by their prime root.



\subsection{An obsolete network stack}
The network TCP/IP \nomenclature{TCP}{Transmission Control Protocol}\nomenclature{IP}{Internet Protocol} stack which is still used today was designed in an era when a few end hosts where static in specific topological positions, communicating over a sole network interface, accessing services like telnet and ftp.
The problems by this approach start to accumulate even in the lowest layers, and specifically the Network Layer.

\paragraph{Network Layer} The Network Layer (or Internet Layer) is responsible for packet forwarding, including routing through intermediate routers, and it does so using a hierarchical IP addressing scheme.
This bind however of a topological-aware IP address to an interface does not manage well with the notion of mobility, where interfaces are not necessarily tied to a specific network.
Nevertheless, an IP address cannot identify forever a host since after a disconnection, the IP address is renewed to one that was most likely previously used by another interface of another machine.
Again, the difficulties in migrating to IPv6 clearly demonstrate the problem with the tight binding of a specific protocol with the programming interfaces (in this case AF\_INET sockets).

\pagebreak
\paragraph{Transport Layer} The Transport Layer provides end-to-end communication services for applications within a layered architecture of network components and protocols.
This is achieved by demultiplexing incoming packets to a socket using the five-tuple (remote IP, remote port, local IP, local port, protocol).
Since local IP is tied to a unique interface, support for migration or multi-path traffic over multiple network interfaces has to be implemented individually by the protocol or the above layers.
Never to forget that every time a renewal of IP address occurs the connection has to be reestablished or at least the other end host has to be notified somehow for the new address. Also, without serving any particular reason, the remote IP address and port have to be exposed to the upper layers. 

For the case of load balancers, every single packet, even from an already established connection, has to pass through them.
This results in a need for dedicated software or hardware, proliferates the demanded computational power, and causes unnecessary ''east-west'' machine-to-machine traffic. In large scale networks with nodes distributed in distant topological locations this can evoke router stretching and increased latency times.

\paragraph{Application Layer}  The Application Layer is an abstraction layer reserved for communication protocols and methods designed for process-to-process communications across an Internet Protocol (IP) computer network.
Because of the overload of IP addresses and ports on lower layers, the Application Layer has to cache them and handle them too.
At the same time, violating the principle of software reuse, each application has to implement from scratch all the logic for the additional functionality of modern Internet (migration, multiple clients support, multihoming, load balancing, mobility etc.), in order to offer it to its users.

Another complication in the Application Layer can be detected during the initiation of a connection, and especially during the mapping of a service identifier to an IP address.
As of now, applications must use out-of-band services like DNS and follow preconcerted conventions before the commencement of the connection.
Additionally, by caching the IP address of the service provider instead or re-resolving the service identifier, the service provider is constrained in changing its IP address (in cases of migration, machine or network failure, multihoming etc.), as it will result to the termination of the established connections and a slow failover, considering that some time is needed for the DNS distributed servers to be updated and to respond correctly to the clients.



\newpage
\subsection{The need for Service-Centric Networking}
In the very early Internet, "calling" the IP address of a machine would get you to one of the killer applications of that time, telnet or ftp.
Those services were run by a single machine and could not accept simultaneous users.
However this approach is not common nowadays, when hundreds of users want to search a keyword in their favorite search engine at the same time.
They do not care about the actual location on the map of the service provider, or which of the machines is serving them accessing a distributed database.
Neither the database of a search engine is that small that can be stored in a single hard disk nor a sole machine can respond to all those requests.
Still such services exist and manage well with the always increasing demand. 

It is only because developers and network administrators are utilizing middleboxes and implementing intermediate systems in order to overcome the deficiencies caused by the superseded network abstractions.
However, this comes with a cost.
Developers have to work with primitive, low-level APIs and to handle many cases of downfalls, needing many costly man-hours, being prone to errors, repeating the same procedure again and again diverging from efficient code writing.
System administrators have to master all those intermediate systems and make them work agreeably.
Routers route packets containing both data and network identifiers without the ability of policy governed delegation.
Replicated service instances run autonomously without a way to directly communicate with each other in a network level.
Master nodes in clusters shoulder the responsibility of the reinstatement over network failures in a wavering manner.
Middleboxes evoke large time delays, they need extra hardware, power, space.
And the list goes on.

To sum up, users nowadays want to access a service or to retrieve some data.
The abstraction of a service can suit well any use of Internet anyone can think of; watch a video, send an e-mail, make a phone call, remotely access a distant machine. Unfortunately so far there is no standardized practice for effectively developing and administering services, abandoning developers to create their own mercurial quick fixes, an expensive, time consuming, inclined to mistakes and complex in orchestration solution.



\newpage
\subsection{A unified control and data plane}
Networking is a constantly developing constituent of the computer science and has played a vast role in its necessity and spreading.
Someone would expect that administering networks is a straight-forward and automated task and innovation scenarios can be easily tested and employed.
However, that's not the case\ldots

Even today, network devices, such as routers, have their forwarding rules (data plane) calculated distributively by protocols (control plane) running on top of them.
This prohibits application developers and network administrators from manipulating the network fabric in a policy-driven manner.
Proprietary technologies, lack of APIs or proper abstractions, non-scalable, inflexible, and troublesome to learn configuration mechanisms, all together stop innovation and agility in network architecture development.

Nevertheless, complexity is added due to the various the discortant developed protocols, which drift slowly in networks with nodes which require multiplicity and dynamism.
It is not rare the case of unreachable nodes, lost packets, big roundtrip times and routing loops until convergence among the devices is succeeded.

Furthermore, the tight bind of the control plane with the network devices makes it impossible to design network-wide management abstractions.
This also impedes any effort on problem detection.
Finally, hosts interface modifications takes a lot of time until it is propagated within the network, dramatically increasing failover times in virtual machines initiations and migrations.



\iffalse
Split those two
\\Proprietary technologies, lack of APIs (programmable interfaces) or proper abstractions, non-scalable, inflexible, and troublesome to learn combine administer
\\Complexity because of many discortant developed protocols
\\Stops innovation and agility in network architecture development
\\Does not cope well with mobile users, server virtualization, cloud services
\\Today’s applications access different databases and servers, creating a flurry of “east-west” machine-to-machine traffic before returning data to the end user device in the classic “north-south” traffic pattern.
\\Require device-level management and manual processes (time! money! availability! errors!)
\\resolve newly observed, constantly arising problems in the current Internet?
\fi
\iffalse
\newpage
\section{Importance of the Problem}
New solutions copy the current stack so changes should be made as soon as possible
\\Necessary benefits for users (multiplicity and dynamism) and for developers (easy, time and money saving, walk through)
\\Administrators must have a better control over the network
\\Need for experimentation
\\Datacenter example
\fi

%%%%%%%%%%%%%%%%%%%%%%%%%%%%%%%%%%%%%%%%%%%%%%%%%%%%%%%%%%%%%%%%%%%%
% Chapter: Service-Centric Networking
\chapterpage{Service-Centric Networking}{SERVICE-CENTRIC\\[0.2cm] NETWORKING}
\newpage
\section{Networked world}
We have far passed the individuality of monads. \emph{We achieve more in groups than we can ever achieve by ourselves. And co-operative interactions are the hallmark of all major evolutionary leap} \cite{Christakis2011}. We rely on our relationships in like manner we slot ourselves within a community. We have communication protocols. We go after passing on our message. We understand that through communicating (from Latin \emph{comm\={u}nic\={a}re}, meaning "to share") with the right intermediaries we finally reach the group of those who can offer the services we want.\\
\indent This is no different from how we understand machine networks.
A eleutheri/euruteri analogy of users and providers, an abstract[TODO] of a client-server architecture.
Because, after all the reason we use networks is to get services.
It is not that we want to connect to a specific machine with 12 (for the moment) dot-separated digits followed by a number in the range of 0-65534.\\
\indent Sometimes we might want to know a specific piece of information; may it be a web page, a video, a JSON document.
Even that request, can well fit in the generalized concept of services.
Therefore content delivery can be theorithei as yet another example of a service \cite{Braun2011}.\\
\indent Nowadays like never before, computer networks are a great resource for finding solutions in our problems.
It would not be a yperboli to defend that we live continuously connected, approvechar services via multiple machines and through different networks.
Even a single mobile device can paradeigmatisei lot to this complexity, with Wifi and 4G interfaces enallasontai breaking ws apotelesma each time the communication channels.\\

\subsection{Modern consumer networks}
512kbps download vs 128kbps upload speed.
This was the killer plan in the beginning of the ADSL era.
At a ratio of 4:1, ISPs \nomenclature{ISP}{Internet Service Provider} had pointed out a basic characteristic of the Internet: \emph{as a user you are expected to consume rather than offer a service or create content}.\\
\indent Reflecting the way to (kapitalistiko) system is working in the real world, 


\subsection{Metropolitan and P2P networks}

\iffalse
\newpage
\section{Service-Centric Networking Principles}
\fi

\subsection{Introduction}
Serval\footnote{More information about the Serval Architecture can be found in the presentation in the Appendix (\ref{sec:servaldhtpres}).} is an end-host stack and a service-centric network architecture, proposed and prototyped by the \href{https://sns.cs.princeton.edu/}{systems and networking group} at \href{https://www.princeton.edu}{Princeton University}, in 2012.

\paragraph{} In the original paper "A Service Access Layer, at Your Service" (2011)\cite{Freedman2011} and later on "Serval: An End-Host Stack for Service-Centric Networking" (2012)\cite{Nordstrom2012}, Nordstr{\"o}m et. al. first decompose the needs of modern networked applications, locate the discordances with the current Network Stack, study previous work and how each of them individually fails to stand as a proper solution, reconsider the current TCP/IP Networking Stack and propose two simple abstractions that can obliterate the legacy problems discussed on Problem Definition\ref{problemdefinition}.\\
\indent Furthermore they introduce a formally-verified end-to-end connection control protocol (ECCP) 

\subsection{Serval Architecture in a few words}


\subsection{Proposed abstractions}



\subsection{Serval Network Stack}
\subsubsection{Service Controller}
\subsubsection{Service Access Layer}
\subsection{Service Resolution in the Serval architecture}

\subsection{Application portability and incremental deployment}
- A translator can be used but: (ADD SCHEMA)
works as an intermediate
slows down performance
apps still work with the traditional apis
 -- adds complexity, translator will have to support many versions/protocols and error proof for them
 -- reimplements the functionality already working on SAL


\subsection{Migrations and incremental deployment}
With Serval being actively under development, it is time to discuss the deployment approaches that could guide us to something that has never happened before; the wide adoption of a new network stack.
Above all, benchmarks prove that tapping Serval in large scale networks as well as datacenters offers a wide range of new functionality in speeds comparable to the original TCP/IP stack ones.\\
\indent A smooth transition to a new architecture requires two things: first that current hardware and intermediary devices are compatible or at least do not interfere with the new packet headers, and that applications are utilizing the late interfaces and are able to dissect and synthesize those packets.

\subsubsection{Legacy Hardware}
SAL's position just on top of the network layer makes it translucent to networking equipment such as hubs, switches and routers, since they are messing up with the headers of up to the network layer.
At this level, hardware is responsible only for delivering the packets to the right destination, the way they have been doing so far.\\
\indent Serval on the contrary is not immune to stateful packet inspection\nomenclature{SPI}{Stateful Packet Inspection} and deep packet inspection\nomenclature{DPI}{Deep Packet Inspection}.
Intermediaries who access the headers of the transport or above layers will have a hard time dissecting a minimum 32 extra bytes following the network layer.
The use of NAT-based agents though, such as load balancers, can be obscured due to the late binding on serviceIDs.
In any way, operation behind legacy networks and NATs can be achieved via UDP encapsulation.

\subsubsection{Modified Programming Interfaces}
Unlike other proposals which can be either integrated in programs as libraries or provide abstractions by overloading identifiers such as ports, Serval's kernel module implementation requires applications to be modified in order to use its active sockets API.\\
\indent In other words, a minimum port of an existing applications would require to include and link to \textless libserval/serval.h\textgreater ~and \textless netinet/serval.h\textgreater ~libraries, set socket family to AF\_SERVAL and substitute system calls to the socket layer such as connect, bind, accept, send etc to use serviceIDs.
Minor modifications might be needed, since new identifiers require different size of bytes to be allocated in memory and so on.\\
\indent The complexity of porting an existing application to Serval depends on how neatly is the connection module isolated.
In general, applications that support various protocols are easier to be ported, since connectivity functions are already decoupled from the program logic and can be replicated to support new APIs.
Large applications, with thousands lines of code,\nomenclature{loc}{lines of code} like Mozilla Firefox require modifications in around just 70 lines of code.
Design patterns like singleton and factory noticeably indicate which parts of the source code should be altered.\\
\indent Nevertheless, applications can simultaneously support multiple protocols and stacks, according to the requests of the other end.
This will be a great advantage in the transitory period of large scale deployment.\\
\indent On the other hand, unexpected runtime results might be confronted due to overlapping of features offered by SAL and reimplemented in the program.
For instance service load balancing at SAL, an inherent component of Serval, might muddle with the application-specific way of allotting requests.
In such cases if possible one of the methods should dominate the final decision.
Either a unique serviceID of an allocated serviceID block --as described in the Hierarchical Resolution approach-- should be used for each instance, and let load-balancing be made by application's functions.
Or load-balancing logic within the application should be removed when a socket's family is AF\_SERVAL and be managed according to service-level rules defined in SAL.\\
\indent We are discussing the lessons learned from the port of nginx\footnote{Nginx [engine-ex] is an Open Source HTTP, reverse proxy and mail proxy server\\ \url{http://nginx.org/}.} to Serval in the following section.


\subsubsection{nginx port to Serval}

Lessons learned from nginx.
There are many 

- Requires to rebuild the whole project with the new abstractions (Serval active sockets) and include Serval libraries

- Depends on the independence of the connectivity modules
if patterns like singleton then porting is easier
calls in many files then complexity arises

- Extra implemented functionality can cause unexpected runtime results
example: 2 levels of loadbalancing


\subsubsection{Incremental Deployment with Serval translator}

- A translator can be used but: (ADD SCHEMA)
works as an intermediate
slows down performance
apps still work with the traditional apis
 -- adds complexity, translator will have to support many versions/protocols and error proof for them
 -- reimplements the functionality already working on SAL

\subsection{Profiling the Serval prototype implementation}
Among the admirable headliners of Serval is the working prototype version of the proposed architecture.
In more than 28000 lines of code\footnote{Source code is open source and can be found in their public repository\\ \url{https://github.com/princeton-sns/serval/}.} covering functionality of the Service Access Layer (both in userlevel operation and as a Linux kernel module), bindings for multiple programming languages, a translator, libraries and examples for writing Serval compatible applications, and with a reported throughput comparable to the unmodified TCP/IP stack, it is clearly showcased the feasibility of the solution.\\
\indent In this section we are profiling the prototype in regard to the following parameters:
\begin{enumerate}
  \item Memory Management
  \item CPU Instructions and Cycles
  \item System Calls times
  \item Execution time needed for the completion of a numbered iteration of requests
  \item The sum of requests that can be satisfied within a given timeframe
\end{enumerate}
Then we will be graphically presenting the results juxtaposed to the measurements of the unchanged TCP/IP stack and the AF\_INET family. The first four tests have not been published before.
\paragraph{} Output was obtained on a ...... machine running Ubuntu 11.04 (Natty Narwahl) kernel version 2.6.38-16-generic (rebuilt with debug symbols). Serval was built with --enable-debug option for the first three tests.\\
For the measurements we ported the nginx\footnote{Nginx [engine-ex] is an Open Source HTTP,reverse proxy and mail proxy server\\ \url{http://nginx.org/}.} web server to Serval architecture, integrating the [nginx\_1.2.9\_serval\_fqdn] branch commits into the build procedure. Also, we implemented a simple HTTP client which supports both AF\_INET and the AF\_SERVAL socket families, depending on the options passed during the call. Source code of both can be found in the Appendix (\ref{sec:sourcecode}).


\iffalse
gprof
perf
google-profile tools
1) memory (oprofile)
2) CPU cycles (callgrind)
3) system calls time (strace)
4) timed execution of 1000 times
5) requests per second
6) Number of packers per request, bytes sent, packet structure
\fi

TODO: Complete this section, maybe add HIP, Chord and OpenFlow


\iffalse
%%%%%%%%%%%%%%%%%%%%%%%%%%%%%%%%%%%%%%%%%%%%%%%%%%%%%%%%%%%%%%%%%%%%
% Chapter: Relative Bibliography
\chapterpage{Relative Bibliography}{RELATIVE BIBLIOGRAPHY}
\section{Review of relative bibliography}
\label{sec:reviewbibliography}
ServalDHT is a multifaceted architecture that combines ideas from a wide spectrum of topics, including but no limited to Large Scale Network Architectures, Network Protocol Layers, Service-Centric Networking, Software Defined Networking, Distributed Hash Table algorithms and security issues of their various implementations, Peer-To-Peer Lookup Services as a replacement to legacy DNS etc.
Therefore references should contain an adequate number of publications on all those themes.\\
\indent The publications that have been used so far as a source of information follow in section 7.
Brief summaries can be found in Appendix at the end of the report.
\subsection{Transport Layer - Decoupling a Host Identity from its location}
\subsubsection{HIP}
\subsubsection{DOA}
\subsubsection{LISP}
\subsubsection{LNA}
\subsubsection{HAIR}
\subsubsection{i3}

\subsection{Application Layer - }
\fi


%%%%%%%%%%%%%%%%%%%%%%%%%%%%%%%%%%%%%%%%%%%%%%%%%%%%%%%%%%%%%%%%%%%%
% Chapter: Service-Aware Networking
\chapterpage{Service-Aware Networking}{SERVICE-AWARE\\[0.2cm]NETWORKING}
\iffalse
\section{Rethinking the Internet experience}
%I HAVE A DREAM
Cloud computing
\\Modularity
\\anonymity, privacy
\\authentication, accountability, encryption, security
\\no middlewares
\\TCP/IP stack data, controller functionality
\\subnetworks
\\mobility and multihoming
\\independent from organizations
\fi



\newpage
\section{Software Defined,\\ Service-centric Networking}
A well thought Service centric architecture provides two things:
\begin{enumerate} \itemsep1pt \parskip0pt 
\parsep0pt
  \item the right abstractions for services, for developers and users to use
  \item a service-aware data/control plane split
\end{enumerate}

\begin{figure}[H]
\centering
\phantomsection
\includegraphics[scale=0.3]{figures/SAL_plane_split}
\caption[Control/Data plane separation in Serval]{Control/Data plane separation in Serval. Data plane service routing roules are stored in the Service table residing in SAL, while control plane logic is implemented in the Service Controller.}
\label{fig:sal_plane_split}
\end{figure}

Confidently, Serval provably does well with both.
This makes us sceptical on if the control plane could be in a way merged with software defined networking APIs.
That way, the network stack could have control over the fabric of the network.
And service-related events could be propagated to SDN-enabled devices as well.

This is the concept behind Service-Aware Networking; 
Software Defined Networking could benefit from Serval's data/control plane separation and enable services running in – possibly distributed – datacenters to automatically, and according to the rights they have been granted, manipulate networks to better utilize the underlying network infrastructure, conforming to their dynamic needs.
In other words, applications using active sockets will be able to modify network topology, co-working with the Service Controller and a centralized SDN controller.
This way, Serval can enable SDN reach the end-host nodes, very much alike SDN-ready applications are currently doing.

A great potential is hidden behind the way service-level routing rules management can be propagated with the use of Service Controllers.
Specifically immediate, network-wide changes are possible with any modification in the Service table or with lower links going up and down.

Actually Serval has in a way its own counterpart software defined API, the \emph{Service Control API}.
One can say so, taking a look in the way Service Controllers are exchanging information with each other and with the Serval Routers.
The integration though with a well-established API, such as Openflow \cite{McKeown2008}, will provide the flexibility of managing OpenFlow switches devices, and writing routing rules on host/interface and service instance changes, in a dynamic policy-driven manner.

\subsection{Scenario: Service Registration as a Network Policy}
Let us imagine a simple case scenario in a datacenter network, which could benefit from the integration of OpenFlow APIs in a Serval-compatible controller.

A POX\footnote{POX is a platform for the rapid development and prototyping of network control software using Python, providing OpenFlow interfaces.\\ \url{http://www.noxrepo.org/pox/about-pox/}} controller is listening for Service Registrations.
Once the instance of a service binds on an active socket with a serviceID, let's say, serviceX, a service registration event is triggered in SAL, a DEMUX rule is added in SAL's service table and the Service Controller is activated.

The Service Controller of that machine is using Serval's Service Control API to notify other Serval-enabled nodes for the new instance.
The Service Controllers of the other machines who receive that message, add a rule for forwarding SYN packets with the destination serviceID equal to serviceX, to the machine which originated the service registration.
If there are other instances of the same service, then the IP address of the machine is added to the relative list of IPs for forwarding.

The Service Controller of the server running the service instance is also informing the SDN controller which is controlling the OpenFlow switches network fabric.
Since the SDN controller has a rule for making serviceX instances available to the outer world, forwarding rules are added in OpenFlow switches for creating paths for packets coming from away the datacenter.

Immediately and without human intervention, as soon as an interface binds on an active socket, it can start listening on requests from clients.
And this functionality is achieved by service-aware networking, mainly with the control-plane mechanisms provided by the service-centric architecture.

A similar case with a virtual machine\nomenclature{VM}{Virtual Machine} migration can be thought.
Serval's in-band signalling can maintain the connection of an active-socket and notify the other end about the new IP address.
Service Controller can inform the SDN controller to create the datapath from external clients to reach the VM with the new IP.
In no time, the VM service instance is ready to serve both existing and new clients, without resetting original sockets state.



\iffalse
\newpage
\section{Service-Aware Networking and Cloud Networks}
How can we take advantage of the above mentioned results in cloud networks?
why cloud fits the description
\fi

\iffalse
\newpage
\section{Introduction to the Proposed Solution}
User Serval as it is, but with a Distributed Service Resolution Service
\\Based on DHT algorithms
\\Run by tier-1 and ISPs, or by users in Autonomous networks
\\Secure identifiers
\\Flat namespace
\\Gets serviceID and
\\- either returns the (IP,Port) back to the client
\\- or forwards the packet directly to the service provider
\\- caches the (serviceID, IP) tuple for future use
\\Incrementally deployable and backwards compatible
\\written in C, running in the user space
\\accept service registration, checks HIP and inform the relative nodes
\\About the Service Controller:
\\will be running as a daemon in the userspace
\\will communicate with ServalDHT to resolve serviceIDs (hashed service names)
\\will enable delegation, load balancing etc as modules (maybe a configuration file?)
\\Draw a FSM for the SRS \nomenclature{SRS}{Service Resolution Service} (http://madebyevan.com/fsm/)
\fi

\newpage
\section{Future Research}
Service-Aware networking is a multi-faceted field, with a lot of research to be done.
We could propose though the continuation of the ideas presented in this thesis with one of the following topics:
\begin{enumerate}
  \item Implement Service Controller (compatible with OpenFlow)
  \item Deployment in PlanetLab or guifinet
  \item Serval router as a virtualized networking service on cisco nexus
  \item Arrakis ("The Operating System is the Control Plane") and Service-Aware networking
  \item ServalDHT
\end{enumerate}

\subsection{ServalDHT: Secure-DHT based Service Resolution Service for the Serval architecture}
\subsubsection{The challenges of private, hierarchical DNS}
Over time, the Internet has gathered a great power over diverse societies.
People all around the world are trusting in order to read about the news, form a political opinion, solve problems in their working environment, do market research.
However, while the Internet continuously affirms its prominent value, it is a surprise how vulnerable it remains to arbitrary (inter)national control and malicious attacks, due to the fact that it is erringly administered by private organizations and its restrained, hierarchical structure. \\
\indent One of the fundamental components of the the functionality of the Internet is the Domain Name System (DNS).
It is used to map human readable names of hosts to numerical IP addresses needed for the purpose of locating service providers around the world and effectively routing traffic to them.
Those identifiers, called domain names, are annually purchased and assigned through the Internet Registry by the private organization ICANN (Internet Corporation for Assigned Names and Numbers).
In addition to being peremptory, this domination also grants to ICANN the privilege of overseeing the content of Internet, by cutting out or declining registration to "undesirable" domains.
Nevertheless, various incidents of catachresis are being observed the last years, with governments like the Egyptian one that shut down its DNS servers to muzzle the protesters in 2011, and the Chinese which still blacklists certain domains as a mechanism for Internet censorship.\\
\indent Moreover, the current Domain Name Service is based on an hierarchical architecture, where (domain name, IP) tuples are cached in midway servers.
This is causing great problems when a service provider has to renew its IP address, because even if a DNS root server is updated, users still get the old cached IP address yet after hours.
This time delay can increase to days in cases where recursive DNS servers do not follow the specified TTL values for their cached entries.
Consequently, hosts are restrained from taking advantage of functionality like multihoming and (virtual machine) migration.
For the same reason, in cases of machine failure, the failover will take a long time, returning in the meanwhile a server unreachable response. \\
\indent Besides, inevitably imitating the hierarchical architecture for domain resolution, autonomous networks must have exclusive, trusted machines offering an analogous service all the time.
This is not always desired, for example in Metropolitan Area Networks (MAN) \nomenclature{MAN}{Metropolitan Area Networks} where ranking does not make sense.\\
\indent Other problems related to DNS bear upon the lack of a widely adopted protocol to correctly verify the real identity of service hosts.
DNS has been proved fallible to various attacks, like (Distributed) Denial of Service attacks (known as DDoS \nomenclature{DDoS}{Distributed Denial of Service} and DoS \nomenclature{DoS}{Denial of Service} attacks), Cache Poisoning (or DNS Spoofing) etc., which aim on deliberately redirecting requests to malevolent hosts.



%%%%%%%%%%%%%%%%%%%%%%%%%%%%%%%%%%%%%%%%%%%%%%%%%%%%%%%%%%%%%%%%%%%%
% Bibliography
\newpage
\phantomsection
\addcontentsline{toc}{part}{Bibliography}
{\Huge \bf \noindent Bibliography}
\nocite{*}
\bibliographystyle{plain}
\renewcommand{\refname}{}
\bibliography{bibliography}


%%%%%%%%%%%%%%%%%%%%%%%%%%%%%%%%%%%%%%%%%%%%%%%%%%%%%%%%%%%%%%%%%%%%
%Appendix
\chapterpage{Appendix}{APPENDIX}
\pagestyle{empty}
\label{sec:appendix}
%%%%%%%%%%%%%%%%%%%%%%%%%%%%%%%%%%%%%%%%%%%%%%%%%%%%%%%%%%%%%%%%%%%%
\newpage
\phantomsection
\addcontentsline{toc}{section}{A. Presentations}
{\Huge \bf \noindent A. PRESENTATIONS}

\newpage
\phantomsection
\addcontentsline{toc}{subsection}{A.1 ServalDHT - Secure DHT based Service Resolution Service}
\label{sec:servaldhtpres}
{\huge \bf \noindent A.1 ServalDHT - Secure DHT\\[0.2cm] based Service Resolution Service}
\includepdf[pages={-}]{ServalDHT-Pres2.pdf}


%%%%%%%%%%%%%%%%%%%%%%%%%%%%%%%%%%%%%%%%%%%%%%%%%%%%%%%%%%%%%%%%%%%%
\newpage
\phantomsection
\addcontentsline{toc}{section}{B. Source Code}
\label{sec:sourcecode}
{\Huge \bf \noindent B. SOURCE CODE}


%%%%%%%%%%%%%%%%%%%%%%%%%%%%%%%%%%%%%%%%%%%%%%%%%%%%%%%%%%%%%%%%%%%%
\newpage
\phantomsection
\addcontentsline{toc}{subsection}{B.1 nginx Serval integration}
\label{sec:nginxport}
{\huge \bf \noindent B.1 nginx Serval integration}\\[0.5cm]
\textbf{File:} ports/nginx/nginx-1.2.9-integrated-serval.patch\\
\textbf{Description:} Integrate Serval patch nginx version 1.2.9 in the build procedure. Based on nginx\_1.2.9\_serval\_fqdn branch commits.\\\\
\textbf{Instructions: }
\begin{enumerate} \itemsep1pt \parskip0pt 
\parsep0pt
	\item Apply the patch using git to nginx\_1.2.9\_serval\_fqdn branch.
	\item Copy libraries from serval/include to a path included in the search path of your compiler (normally that should be /usr/local/include/)
	\item Configure nginx with serval (ports/nginx/configure --with-serval)
	\item Make sure configure script found netinet/serval.h library and can build AF\_SERVAL
	\item If everything is fine, you should be able to see "+ using serval active sockets" in the configuration summary
	\item Compile with make and then execute make install
	\item Edit nginx configuration file (by default /usr/local/nginx/conf/nginx.conf) and uncomment the virtual host configured for the serval architecture
	\item Restart nginx and now you can accept both AF\_INET and AF\_SERVAL requests
	\item nginx Service ID is 8\\[0.5cm]
\end{enumerate}
\lstinputlisting[language=diff]{source_code/nginx-serval-build-patch.diff}

%%%%%%%%%%%%%%%%%%%%%%%%%%%%%%%%%%%%%%%%%%%%%%%%%%%%%%%%%%%%%%%%%%%%
\newpage
\phantomsection
\addcontentsline{toc}{subsection}{B.2 Wireshark Lua Serval Dissector}
{\huge \bf \noindent B.2 Wireshark Lua Serval Dissector}\\[0.5cm]
\textbf{File:} serval/src/serval\_wireshark\_dissector.lua\\
\textbf{Description:} Wireshark dissector for the Serval protocol (IPPROTO\_SERVAL 144).\\
\textbf{Instructions: }
\begin{enumerate} \itemsep1pt \parskip0pt 
\parsep0pt
	\item Make sure Lua is enabled in wireshark's global configuration (\href{http://wiki.wireshark.org/Lua}{Wireshark wiki})
	\item Copy serval/src/serval\_wireshark\_dissector.lua into a plugin directory (default is $\sim$/.wireshark/plugin)\\[0.5cm]
\end{enumerate}
\lstinputlisting[language=lua]{source_code/serval_wireshark_dissector.lua}

%%%%%%%%%%%%%%%%%%%%%%%%%%%%%%%%%%%%%%%%%%%%%%%%%%%%%%%%%%%%%%%%%%%%
\newpage
\phantomsection
\addcontentsline{toc}{subsection}{B.3 Initialize Serval test node Script}
{\huge \bf \noindent B.3 Initialize Serval test node Script}\\[0.5cm]
\textbf{File:} serval/src/init\_servtest.sh\\
\textbf{Description:} Initialize a Serval testing node.\\
\textbf{Instructions: }
\begin{enumerate} \itemsep1pt \parskip0pt 
\parsep0pt
	\item You might want to set the executable bit (chmod +x init\_servtest.sh)
	\item Execute the script with superuser privileges
	\item View help with \texttt{-}h\\[0.5cm]
\end{enumerate}
\lstinputlisting[style=BashColor]{source_code/init_servtest.sh}

%%%%%%%%%%%%%%%%%%%%%%%%%%%%%%%%%%%%%%%%%%%%%%%%%%%%%%%%%%%%%%%%%%%%
\newpage
\phantomsection
\addcontentsline{toc}{subsection}{B.4 HTTP Client}
{\huge \bf \noindent B.4 HTTP Client}\\[0.5cm]
\textbf{File:} serval/src/test/http\_client.c\\
\textbf{Description:} HTTP client that works with both AF\_SERVAL and AF\_INET families.\\
\textbf{Instructions: } Print usage with \texttt{-}h or \texttt{{-}{-}}help\\[0.5cm]
\lstinputlisting[style=CColor]{source_code/http_client.c}

%%%%%%%%%%%%%%%%%%%%%%%%%%%%%%%%%%%%%%%%%%%%%%%%%%%%%%%%%%%%%%%%%%%%
\newpage
\phantomsection
\addcontentsline{toc}{subsection}{B.5 libmicrohttpd Serval port}
{\huge \bf \noindent B.5 libmicrohttpd Serval port}\\[0.5cm]
\textbf{File:} ports/libmicrohttpd-0.9.36/libmicrohttpd\_serval.patch\\
\textbf{Description:} Patch libmicrohttpd version 0.9.36 to use Serval active sockets. Libmicrohttpd is a free, small C library that can be included in C and C++ applications and provide HTTP server functionality.\\
\url{https://www.gnu.org/software/libmicrohttpd/}\\
\textbf{Instructions: }
\begin{enumerate} \itemsep1pt \parskip0pt 
\parsep0pt
	\item Include the library in any application, usually as a daemon.
	\item As a reference use the boilerplates at:\\
	ports/libmicrohttpd-0.9.36/src/examples
	\item You might want to test the src/examples/minimal\_example (give as a parameter a random port). It's serviceID is hardcoded and equal to htonl(80), so it can be used with http\_client out of the box.\\[0.5cm]
\end{enumerate}
\lstinputlisting[language=diff]{source_code/libmicrohttpd_serval.patch}

\newpage
\phantomsection
\vspace*{\fill}
\hrulefill
\begin{center}
Isaakidis Marios -- misaakidis@yahoo.gr\\
Cyprus University of Technology, 2014
\end{center}
\end{document}