\newpage
\section{Networked world}
We have far passed the individuality of monads.
We achieve more in groups than we can ever achieve by ourselves.
And co-operative interactions are the hallmark of all major evolutionary leap \cite{Christakis2011}.
We rely on our relationships in like manner we slot ourselves within a community.
We have communication protocols.
We go after passing on our message.
We understand that through communicating (from Latin \emph{comm\={u}nic\={a}re}, meaning "to share") with the right intermediaries we finally reach the group of those who can offer the services we want.
\\ \indent This is no different from how we understand machine networks.
A broader analogy of users and providers, an abstract representations of the client-server architecture.
Because, after all, the reason we use networks is to get services.
It is not that we want to connect to a specific machine with 12 (for the moment) dot-separated digits followed by a number in the range of 1-65535.
\\ \indent Sometimes we might want to know a specific piece of information; may it be a web page, a video, a JSON document.
Even that request, can well fit in the generalized concept of services.
Therefore content delivery can be considered as yet another example of a service \cite{Braun2011}.
\\ \indent Nowadays like never before, computer networks are a great resource for finding solutions in our problems.
It would not be an exaggeration to defend that we live continuously connected, redeeming services via multiple machines and through different networks.
Even a single mobile device can exemplify this complexity, with Wifi and 4G interfaces taking turns, breaking as a result each time the communication channels.

\subsection{Modern consumer networks}
512kbps download vs 128kbps upload speed.
This was the killer plan in the beginning of the ADSL era.
At a ratio of 4:1, ISPs \nomenclature{ISP}{Internet Service Provider} had pointed out a basic characteristic of the Internet: \emph{as a user you are expected to consume rather than offer a service or create content}.
\\ \indent Reflecting the way consumption economics is working in the real world, in the Internet it is easy to spot right on the huge service providers who circulate network services today.
In an average day, more than 60 percent of all Web-enabled devices exchange traffic with Google's servers \footnote{{\url{http://www.cnet.com/news/google-sets-internet-record-with-25-percent-of-u-s-traffic/}, accessed on May 2014}}.
Likewise, Netflix video streaming provider, which represents one third of all down-stream traffic \footnote{\url{http://www.hollywoodreporter.com/news/video-accounts-53-percent-internet-655203}, accessed on May 2014}.
\\ \indent Still, 


\subsection{Metropolitan and P2P networks}
sharing economy 
virtual currencies
crypto-currencies like bitcoin
Bittorrent only 4\% of global traffic