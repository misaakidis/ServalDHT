\subsection{Service name Resolution}
Service names by definition are human readable strings similar to domain names, for identifying services.
Consequently they are in a format that is incompatible to Serval active sockets API.
A method is needed, for mapping those hexadecimal strings to serviceIDs, so they can traverse the network in the service-level routing process described in the previous section.

Serval does not dictate the way service names are resolved to service instances.
In this section we discuss the different approaches in service resolution, a crucial component of any architecture that aims to be adopted in a wide scale.

What we consider important highlighting, is the belief that not always all clients will want to be part of the same service resolution scheme.
In the beginning of the chapter, we discussed the specificities of consumer and peer-to-peer networks.
Based on today's Internet status, it is safe to assume that the hierarchical resolution model with managing authorities could prevail.
But we can be sure that independent, flat resolution services will emerge in a completely separate namespace, over the same or different network wiring.
Still, developers are the ones who will make the final decision on how client applications resolve a service name and reach a service instance, much alike the way it is now.



\subsubsection{Resolution based on a priori knowledge}
Like in the beginning of the Internet, for the moment Serval applications use hardcoded serviceIDs to reach a service.
Violating in a way the service name abstraction, in this transitional stage, serviceIDs are either defined as constants or passed to the program as arguments.
A preordained hashing algorithm and other conventions are used for orchestrating the service registration and resolution, defining in a manner the namespace bounds.

This technique can also be useful in ad hoc (computer-to-computer) networks, when a client needs to reach a service and there are no service resolution services available.
As well as for supporting bootstrapping in other resolution schemes --in the same way as resolv.conf and port 53 are used now, or a list of nodes is pre-fetched in torrent (DHT) applications.


\subsubsection{Hierarchical Resolution}
\label{sec:hierresol}
Much like to the current domain name system domains \nomenclature{URI}{Uniform Resource Locator}, serviceIDs could be managed by an authoritative organization.
This fits well with serviceIDs' potentiality of being allocated as blocks, to organizations and individuals to manage in a longest-prefix match way.

We can imagine a service similar to resolver, mapping a service name to a serviceID.
Lookups will be forwarded recursively to a greater level if they cannot be answered.
Different hierarchical resolution servers might exist and clients should be able to select which one to use by default.
Network administrators might want to setup private service name resolvers, for caching functionality, blocking specific service names (or a block of serviceIDs), or adding new service name to serviceID relations.



\subsubsection{Flat Resolution Schemes}
Another approach suggested by the authors is that serviceIDs get calculated using a hash function on a service prefix and an application-level unique key.
That key, could be content-specific as well.
For example if the application is a web server, prefix bytes indicate that the service is a web server and the key could be a blog name and so on.

Moreover, if the application-level key is omitted, supporting a completely \emph{semantic-free, flat} namespace is also possible.
By semantic-free\cite{Walfisha2004} reference (SFR\nomenclature{SFR}{Semantic Free Reference}) we mean that service identifiers are completely independent from topological, network, or any other characteristic of the service or the provider.

Likewise torrent DHT networks, in order to be able to become a node in the ring, bootstrapping is required with a list of permanent nodes.
Flat resolution schemes, though not suggested for Internet service resolution, they can be useful when it comes to networks without a hierarchical structure, like datacenter and Wide Area networks.